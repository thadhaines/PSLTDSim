\documentclass[12pt]{article}
\usepackage[english]{babel}
\usepackage[utf8]{inputenc}

%% Pointer to 'default' preamble
\input{../../../../ResearchDocs/TEX/thad_preamble.tex}

%% Header
\rhead{Thad Haines \\ Page \thepage\ of \pageref{LastPage}}
\chead{MiniWECC\\ Deviation Plots Introduced}
\lhead{Research \\ }

\begin{document}
	\paragraph{Deviation Plots:} To make large numbers of comparisons easier to understand, deviation plots were created to show the difference between LTD and PSDS data.\\
	
	One way to think of these plots is $\text{LTD}_{data}+\text{Deviation}_{data} = \text{PSDS}_{data}$. (Assuming all time step issues are handled appropriately.)\\
	
	Alternatively, the deviation data could be thought of as data that is filtered out due to the larger time steps and assumptions made by LTD.\\
	
	\newcommand{\caseName}{miniWECCgenTrip027}
	\includegraphics[width=\linewidth]{\caseName dev1} 
	
	\includegraphics[width=.5\linewidth]{\caseName Pe2} %
	\includegraphics[width=.5\linewidth]{\caseName Pm2} 
	
	\includegraphics[width=.5\linewidth]{\caseName Q2} %
	\includegraphics[width=.5\linewidth]{\caseName V2} 
	
	Angle deviation not included as PSDS angles wrap oddly and would result in a misleading deviation.
\end{document}